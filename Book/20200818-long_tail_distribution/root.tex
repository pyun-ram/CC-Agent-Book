\documentclass{article}

\title{Long-tail distribution}
\author{Peng YUN}
\date{20200818}

\begin{document}
    \maketitle
\section{A general description of the long-tail distribution}
The natural occurrence of classes in the world is a long
tailed distribution \cite{Zhu_2014_CVPR,van2017devil,Wang2017tail},
whereby instances for most classes are rare and instances for few classes are abundant.

\section{Another general description of the long-tail distribution}
"It should be noted, however, that even when one has an apparently massive data set, the effective number of data points for certain cases of interest might be quite small. In fact, data across a variety of domains exhibits a property known as the \textbf{long tail}, which means that a few things (e.g. words) are very common, but most things are quite rare. For example, 20\% of Google searches each day have never been seen before\cite{murphy2012machine}."
\bibliographystyle{unsrt}
\bibliography{root}
\end{document}